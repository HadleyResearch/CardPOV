% File: project_notebook.tex
% Description: TeX file to generate Project Notebook (template)
% Author: George Hadley
% Website: http://nbitwonder.com
% Notes:
% 1) This document is written using the LaTeX typesetting language. For more information on
%	LaTeX, consult http://en.wikibooks.org/wiki/LaTeX/
% 2) This document needs to be compiled using pdfLaTeX. It is not supported with pdfTeX
%	at the present time
% 3) This document utilizes the \nbwheader command created in doc_header.tex. By default,
%	this file is located at:
%	/path-to-documentation-templates/lbr/doc_header.tex
% 4) This document utilizes commands and environments created in projnb_lbr.tex. By default,
%	this file is located at:
%	/path-to-documentation-templates/lbr/projnb_lbr.tex
% Version: 0.1
% Last Modified: 1-04-2010
\documentclass[12pt,letterpaper,onecolumn]{article}
\usepackage{graphicx}
\usepackage{float}
\usepackage{subfig}
\usepackage{tikz}
\usepackage{fancyhdr}
\usepackage{kpfonts}  %Uncomment if kpfonts is installed and you want to use a non-default font
\usepackage{verbatim}
\usepackage{fullpage}
\usepackage{hyperref}

%Path to global documentation library functions
%Modify this to /path-to-documentation-templates/lbr
\newcommand{\globallbr}{../lbr}

%Page layout settings
\setlength{\voffset}{-10pt}
\setlength{\headsep}{20pt}
\setlength{\headheight}{15pt}
\setlength{\topmargin}{-20pt}

\begin{comment}
  Hyperref settings: settings for the hyperref hyperlink package.
  For a more detailed listing of available settings, consult 
  	http://en.wikibooks.org/wiki/LaTeX/Hyperlinks#Customization

  IMPORTANT: For your document, modify the pdftitle, pdfauthor, pdfsubject,
	and pdfkeywords options to tailor to your document
\end{comment}
\hypersetup{
	bookmarks=true, 					%Enable pdf bookmarks
	pdfborder={0,0,0},					%Disable borders around links
	pdftitle={Project Notebook (template)},		%Name of PDF document
	pdfauthor={George Hadley},				%Author of PDF document
	pdfsubject={Embedded Electronics},		%Subject of PDF document
	pdfkeywords={diy,electronics,nbitwonder},	%Keywords for PDF document
	colorlinks={true},					%Enable colored links
	linkcolor=red,						%Internal link color
	citecolor=green,						%Citation link color
	filecolor=blue,						%File link color
	urlcolor=blue						%URL link color
}
\input{\globallbr/doc_header.tex}
\input{\globallbr/aliases.tex}
\input{lbr/project_notebook_lbr.tex}

%Modify these to reflect your documentation
\newcommand{\documentationtype}{Project Notebook}
\newcommand{\projectname}{CardPOV  }
\newcommand{\projectversion}{1.0  }

%Header/Footer Definitions
\pagestyle{fancy}
\lhead{ }
\chead{ }
\rhead{\projectname  v\projectversion  \documentationtype}
\lfoot{\href{http://nbitwonder.com}{http://nbitwonder.com} }
\cfoot{\thepage}
\rfoot{\copyright 2011 NBitWonder, LLC}

\begin{document}
\thispagestyle{plain}
% Insert title page or header here
\nbwheader{\documentationtype}{\projectname}{\projectversion}
% Insert optional project index here
% For long projects, it is recommended that project entries be indexed
%	by entry, week, month, or year (or a combination of the above)

% Project Notebook entries
% Project Notebook entries should be enclosed in boxes and contain a tagline
%	detailing the entry number, date, revision (project minor version), and 
%	a short, descriptive title
\begin{nbentry}{001}{10/19/2009}{1.00}{Initial Design}
\underline{Hardware Section:} \\
\textbf{Microcontroller:} A microcontroller with onboard USB and sufficient memory space for future code
	features is desired. For this, the \textbf{PIC18F25J50}, featuring 32k of flash and 2k of data memory,
	as well as onboard USB, will be used. \\
\textbf{Power Source:} A slim power source with sufficient capacity to run the onboard circuitry is desired.
	For this, a \textbf{CR2032} coin cell battery, providing 3V power and a 230mAh capacity, will be used. \\
\textbf{LEDs:} Generic surface mount LEDs \\
\textbf{Switches:} Generic tactile switches
\hfill\newline\hfill\newline
\noindent\underline{Firmware:}
Version 1 will make use primarily of the timer0 peripheral to coordinate the LED pulses. For static timing
	purposes, it will be assumed that the POV will be moved repetitively through a linear path approximately
	30" in length. POV characters will be stored in the onboard data EEPROM memory. Target frequency of
	operation is 8MHz.
\hfill\newline\hfill\newline
\noindent\underline{Pin Usage:}
Of particular importance for the initial layout of the CardPOV, pins should be selected in such a way so as to
	not conflict with future pin needs. For this reason, pins that are needed for the USB, comparator, and
	ADC should not be used. Initial pin declarations are as follows:
	\hfill\newline\hfill\newline
	\noindent
	Pushbutton 1: RC0 (pin 11) \\
	Pushbutton 2 (available): RC1 (pin 12) \\
	Pushbutton 3 (available): RC2 (pin 13) \\
	POV LEDs: RB5-RB0, RC7, RC6 (pins 26-21, 18, 17)
\end{nbentry}

\begin{nbentry}{002}{1/11/2010}{1.00}{Ordering and PCB Errors}
Work continued on CardPOV v1 yesterday, but was quickly brought to a halt when the following problems
arose:
\begin{enumerate}
\item An ordering error resulted in the accidental purchase of PIC18LF25J50 microcontrollers instead of 
PIC18F25J50 microcontrollers
\item A second ordering error resulted in the accidental procurement of 0.75\ohm resistors instead of 
75\ohm resistors
\item In the PCB layout, ground was referred to by 2 different names (GND and VSS), and as such these 2 nets
were never connected, resulting in incomplete circuit connections to the LEDs.
\end{enumerate}
To help mitigate these problems in the future, I plan to check orders and PCB layouts more carefully.
\end{nbentry}

% Sources Cited
% Add a section detailing all external source citations. The IEEE citation
%	style is recommended. For guidelines on using the IEEE citation style,
%	refer to
%   http://www.ieee.org/portal/cms_docs_iportals/iportals/publications/authors/transjnl/stylemanual.pdf

\end{document}
